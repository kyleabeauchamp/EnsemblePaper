\documentclass[11pt,titlepage]{article}
\usepackage[utf8]{inputenc}
%\usepackage[margin=1.0in]{geometry}
\usepackage{graphicx}
\usepackage{subfigure}
\usepackage{amssymb,amsfonts,amsmath}
\usepackage{url}
\usepackage{booktabs, multicol, multirow}

\title{}

\begin{document}


\begin{abstract}

Inferring biomolecular conformation from experiment is a fundamental goal of structural biology.  Structure determination often requires the combination of modeling and experiment, but the vast majority of approaches model only a single conformation, provide limited uncertainty information, and inherit biases from assumed force fields when data are limited.  Building on recent conceptual advances, we hypothesized that these biases and missing uncertainty estimates could be addressed through Bayesian Energy Landscape Tilting (BELT), a scheme that enables the systematic computation of fully Bayesian 'hyperensembles' over conformational ensembles.  As a test of BELT's ability to correct force field bias, we show that conformational ensembles of trialanine derived from five different force fields (ff96, ff99, ff99sbnmr-ildn, CHARMM27, and OPLS-AA) and chemical shift and scalar coupling measurements give convergent values of the peptide's $\alpha$, $\beta$, and $PP_{II}$ conformational populations. Furthermore, 
the ensembles recover set-aside measurements not used in the fitting. BELT's principled combination of simulation and limited experimental data promises rigorous assessment of force field bias and sets a foundation for modeling ensembles and uncertainties in complex biomolecular systems.   

\end{abstract}



\end{document}